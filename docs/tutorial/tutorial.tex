\section{Lua за 15 минут}

\lstset{style=Lua}

Данное пособие является адаптацией статьи ``Learn Lua in 15 Minutes''. 
Оригинал на английском языке можно найти по адресу: \url{http://tylerneylon.com/a/learn-lua/}.

\subsection{Комментарии}

Комментарии в Lua можно сделать двумя способами:

\begin{lstlisting}
-- One line comment
	
--[[
    first line
    seconde line
--]]
\end{lstlisting}

Первый способ начинает однострочный комментарий, второй~---~многострочный.

\subsection{Переменные}

Все числовые переменные являются вещественными (double):

\begin{lstlisting}
number = 42
another_number = 3.1415
\end{lstlisting}

Строки в языка Lua являются \emph{неизменяемыми}, то есть нельзя обратиться к индексу строки и поменять символ.
Объявление сток можно сделать тремя способами:

\begin{lstlisting}
color = 'black'
season = "summer"
huge_string = [[ This is
	a very-very
	long string! ]]
\end{lstlisting}

Переменные могут принимать логическое значение \emph{boolean}: \textbf{true} (истина) или \textbf{ложь}:

\begin{lstlisting}
to_be_or_not_to_be = true
\end{lstlisting}

Переменные также могут принимать значение \emph{nil}. Данный тип означает, что значения у переменной \textbf{не существует}!

\begin{lstlisting}
aliens_exist = nil
\end{lstlisting}
