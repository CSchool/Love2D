\section{Lua за 15 минут}

\lstset{style=Lua}

Данное пособие является адаптацией статьи ``Learn Lua in 15 Minutes'' с некоторыми дополнениями. 
Оригинал на английском языке можно найти по адресу: \url{http://tylerneylon.com/a/learn-lua/}.

\subsection{Комментарии}

Комментарии в Lua можно сделать двумя способами:

\begin{lstlisting}
-- One line comment
	
--[[
    first line
    seconde line
--]]
\end{lstlisting}

Первый способ начинает однострочный комментарий, второй~---~многострочный.

\subsection{Переменные (простые типы)}

Все числовые переменные являются вещественными (double):

\begin{lstlisting}
number = 42
another_number = 3.1415
\end{lstlisting}

Над числами можно проводить следующие операции: сложение ($+$), вычитание ($-$), умножение ($*$), деление ($/$), возведение в  степень (\^{}):

\begin{lstlisting}
add = 5 + 3 -- add = 8
sub = add - 4 -- sub = 4
mult = add * sub -- mult = 32
div = add / sub -- div = 2
pow = 2^3 -- pow = 8
\end{lstlisting}

Строки в языка Lua являются \emph{неизменяемыми}, то есть нельзя обратиться к индексу строки и поменять символ.
Объявление сток можно сделать тремя способами:

\begin{lstlisting}
color = 'black'
season = "summer"
huge_string = [[ This is
	a very-very
	long string! ]]
\end{lstlisting}

Для соединения строк (\emph{конкатенация} строк) используется оператор $..$:

\begin{lstlisting}
name = "Petr"
surname = "Ivanov"
pupil = name .. " " .. surname -- pupil = "Petr Ivanov"
\end{lstlisting}

Если при конкатенации строк будут использоваться числовые переменные, то они автоматически будут приведены к строкам:

\begin{lstlisting}
number = 42
question = number .. " is good answer for everything!"
-- question = "42 is good answer for everything!"
\end{lstlisting}

Переменные могут принимать логическое значение \emph{boolean}: \textbf{true} (истина) или \textbf{ложь}:

\begin{lstlisting}
to_be_or_not_to_be = true
\end{lstlisting}

Переменные также могут принимать значение \emph{nil}. Данный тип означает, что значения у переменной \textbf{не существует}!

\begin{lstlisting}
aliens_exist = nil
\end{lstlisting}

\subsection{Операторы отношений}

В языке Lua выделяются следующие операторы отношений, каждый из которых возвращает \emph{true} или \emph{false}:

\begin{lstlisting}
<   >   <=  >=  ==  ~=
\end{lstlisting}

Оператор $==$ проверяет равенство аргументов, а оператор \lstinline{~=}~---~неравенство:

\begin{lstlisting}
print(5 == 6) -- false
print(52 ~= 0) -- true
\end{lstlisting}

\subsection{Условный оператор if}

Условия в языке Lua записываются при помощи условного оператора \emph{if}:

\begin{lstlisting}
if statement then
... -- do something if statement == true
end
\end{lstlisting}

Оператор проверяет условие \emph{statement} и выполняет операции между ключевыми словами \emph{then} и \emph{end} только в том случае, если \emph{statement}~---~истинен.

Примеры условий:

\begin{lstlisting}
if a<0 then a = 0 end

if object == "car" then
  print("This is car!")
end
\end{lstlisting}

Можно задавать поведение условного оператора \emph{if} при помощи ключевого слова \emph{else}, в случае, если условие \emph{statement}~---~ложно:

\begin{lstlisting}
if statement then
... -- statement == true
else
... -- statement == false
end
\end{lstlisting}

Пример использования:

\begin{lstlisting}
if age < 18 then
  print("You can't go to this movie!")
else
  print("Your age is allowed for this movie")
end
\end{lstlisting}

Иногда могут понадобится для работы множественные ветвления (\emph{elseif}) условного оператора \emph{if}:

\begin{lstlisting}
if op == "+" then
  r = a + b
elseif op == "-" then
  r = a - b
elseif op == "*" then
  r = a*b
elseif op == "/" then
  r = a/b
else
  print("Error!")
end
\end{lstlisting}

Отрицание логического выражения \emph{statement} задается при помощи ключевого слова \emph{not}:

\begin{lstlisting}
if not end_of_game then ... end
\end{lstlisting}

