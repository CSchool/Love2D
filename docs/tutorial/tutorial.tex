\section{Lua за 15 минут}

\lstset{style=Lua}

Данное пособие является адаптацией статьи ``Learn Lua in 15 Minutes'' с некоторыми дополнениями. 
Оригинал на английском языке можно найти по адресу: \url{http://tylerneylon.com/a/learn-lua/}.

\subsection{Комментарии}

Комментарии в Lua можно сделать двумя способами:

\begin{lstlisting}
-- One line comment
	
--[[
    first line
    seconde line
--]]
\end{lstlisting}

Первый способ начинает однострочный комментарий, второй~---~многострочный.

\subsection{Переменные (простые типы)}

Все числовые переменные являются вещественными (double):

\begin{lstlisting}
number = 42
another_number = 3.1415
\end{lstlisting}

Над числами можно проводить следующие операции: сложение (\lstinline{+}), вычитание (\lstinline{-}), 
умножение (\lstinline{*}), деление (\lstinline{/}), возведение в  степень (\lstinline{^}):

\begin{lstlisting}
add = 5 + 3 -- add = 8
sub = add - 4 -- sub = 4
mult = add * sub -- mult = 32
div = add / sub -- div = 2
pow = 2^3 -- pow = 8
\end{lstlisting}

Строки в языка Lua являются \emph{неизменяемыми}, то есть нельзя обратиться к индексу строки и поменять символ.
Объявление сток можно сделать тремя способами:

\begin{lstlisting}
color = 'black'
season = "summer"
huge_string = [[ This is
	a very-very
	long string! ]]
\end{lstlisting}

Для соединения строк (\emph{конкатенация} строк) используется оператор $..$:

\begin{lstlisting}
name = "Petr"
surname = "Ivanov"
pupil = name .. " " .. surname -- pupil = "Petr Ivanov"
\end{lstlisting}

Если при конкатенации строк будут использоваться числовые переменные, то они автоматически будут приведены к строкам:

\begin{lstlisting}
number = 42
question = number .. " is good answer for everything!"
-- question = "42 is good answer for everything!"
\end{lstlisting}

Переменные могут принимать логическое значение \emph{boolean}: \textbf{true} (истина) или \textbf{ложь}:

\begin{lstlisting}
to_be_or_not_to_be = true
\end{lstlisting}

Переменные также могут принимать значение \emph{nil}. Данный тип означает, что значения у переменной \textbf{не существует}!

\begin{lstlisting}
aliens_exist = nil
\end{lstlisting}

\subsection{Логические операторы}

Существуют следующие логические операторы: \lstinline{and}, \lstinline{or} и \lstinline{not}.
Все логические операторы предполагают, то \lstinline{false} и \lstinline{nil} представляют собой значение \textbf{false}, 
а все остальные значения~---~\textbf{true}.

Оператор \lstinline{and} возвращает первый аргумент в том случае, если его значение \emph{false}, 
в противном случае возвращается второй аргумент.
Оператор \lstinline{or} возвращает  первый аргумент в том случае, если его значение \emph{true},
в противном случае возвращается второй аргумент.

\begin{lstlisting}
print(4 and 5)         -- 5
print(nil and 13)      -- nil
print(false and 13)    -- false
print(4 or 5)          -- 4
print(false or 5)      -- 5
\end{lstlisting}

Операторы \lstinline{and} и \lstinline{or} не вычисляют второй аргумент, если в это нет необходимости.
Например, выражение \lstinline{x = x or v} эквивалентно следующему выражению:

\begin{lstlisting}
if not x then x = v end
\end{lstlisting}

То есть, если значение \lstinline{x} не существует, то ставится значение \lstinline{v}.

Ещё один вариант использования условных операторов: реализация тернарного оператора (\lstinline{a ? b : c}). В языке Lua его можно реализовать следующим способом:

\begin{lstlisting}
a and b or c -- (a and b) or c
\end{lstlisting}

Пример выбора максимального значения из двух чисел:

\begin{lstlisting}
max = (x > y) and x or y
\end{lstlisting}

Сперва вычисляется выражение \lstinline{x > y}. Если оно имеет значение \lstinline{true}, то срабатывает \lstinline{(x > y)    and x} и возвращается \lstinline{x}, так как \lstinline{x}~---~число и всегда равен значению \lstinline{true}. 
Если же выражение  \lstinline{x > y} имеет значение \lstinline{false}, 
то выражение \lstinline{(x > y)   and x} возвращает \lstinline{false}, оно сравнивается с \lstinline{y}, 
и оператор \lstinline{or} возвращает значение \lstinline{y}.

Оператор \lstinline{not} всегда возвращает \lstinline{true} или \lstinline{false}:

\begin{lstlisting}
print(not nil)      -- true
print(not false)    -- true
print(not 0)        -- false
print(not not nil)  -- false
\end{lstlisting}

\subsection{Операторы отношений}

В языке Lua выделяются следующие операторы отношений, каждый из которых возвращает \lstinline{true} или \lstinline{false}:

\begin{lstlisting}
<   >   <=  >=  ==  ~=
\end{lstlisting}

Оператор $==$ проверяет равенство аргументов, а оператор \lstinline{~=}~---~неравенство:

\begin{lstlisting}
print(5 == 6) -- false
print(52 ~= 0) -- true
\end{lstlisting}

\subsection{Условный оператор if}

Условия в языке Lua записываются при помощи условного оператора \lstinline{if}:

\begin{lstlisting}
if statement then
... -- do something if statement == true
end
\end{lstlisting}

Оператор проверяет условие \emph{statement} и выполняет операции между ключевыми словами \lstinline{then} и \lstinline{end} только в том случае, если \emph{statement}~---~истинен.

Примеры условий:

\begin{lstlisting}
if a < 0 then a = 0 end

if object == "car" then
  print("This is car!")
end
\end{lstlisting}

Можно задавать поведение условного оператора \lstinline{if} при помощи ключевого слова \lstinline{else}, в случае, если условие \emph{statement}~---~ложно:

\begin{lstlisting}
if statement then
... -- statement == true
else
... -- statement == false
end
\end{lstlisting}

Пример использования:

\begin{lstlisting}
if age < 18 then
  print("You can't go to this movie!")
else
  print("Your age is allowed for this movie")
end
\end{lstlisting}

Иногда могут понадобится для работы множественные ветвления (\lstinline{elseif}) условного оператора \lstinline{if}:

\begin{lstlisting}
if op == "+" then
  r = a + b
elseif op == "-" then
  r = a - b
elseif op == "*" then
  r = a*b
elseif op == "/" then
  r = a/b
else
  print("Error!")
end
\end{lstlisting}

Отрицание логического выражения \emph{statement} задается при помощи ключевого слова \lstinline{not}:

\begin{lstlisting}
if not end_of_game then ... end
\end{lstlisting}

Выражение \emph{statement} может содержать в себе сложные логические выражения:

\begin{lstlisting}
if age >= 14 and age <= 18 then ... end
\end{lstlisting}
