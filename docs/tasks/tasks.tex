\lstset{style=Lua}


\section*{Переменные}

Считывание с консоли можно сделать при помощи \lstinline{io.read()}, привести к числу~---~\lstinline{tonumber()}:

\begin{lstlisting}
local a = 0
a = tonumber(io.read())
print(a)
\end{lstlisting}

\textbf{Список задач:}

\begin{enumerate}
	\item Создать две переменные, cчитать числа, вывести их на экран;
	\item Считать два числа, вывести результаты арифметических операций (\lstinline{+}, \lstinline{-}, \lstinline{*}, \lstinline{/}, \lstinline{%}, \lstinline{^})
\end{enumerate}

\section*{Условия}

\textbf{Список задач:}

\begin{enumerate}
	\item Считать три числа, вывести наибольшее из них;
	\item Считать число, вывести сообщение о том, четное ли оно.
\end{enumerate}

\section*{Циклы}

\begin{enumerate}
	\item Вывести числа от 1 до 100 c шагом 3 при помощи циклов (реализовать 3 варианта решения данной задачи);
	\item Вывести все нечетные числа в диапазоне от $a$ до $b$ (переменные считываются с клавиатуры);
	\item Вывести все возможные ``Счастливые билетики'' (сумма первых трех цифр равна сумме последних трех цифр).
\end{enumerate}

\section*{Таблицы}

\begin{enumerate}
	\item Создать массив, заполнить его значениями и вывести на экран в формате ``'индекс'~---~``значение'';
	\item Создать таблицу, в которой можно хранить информацию про детей лагеря ``Зарница''~---~хранить имя, фамилию, номер отряда. Создать массив этих таблиц, заполнить их и вывести чей отряд является самым многочисленным.
\end{enumerate}